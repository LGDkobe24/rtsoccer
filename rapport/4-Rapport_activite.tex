\chapter{Rapport d'activité}



\section{Méthodes de travail}

Après la validation de notre projet en tant que sujet de TER, nous avons commencé par nous affecter des rôles autre que développeur (puisque nous le sommes tous):
\begin{itemize}
        \item Bricas Samuel : Chef de projet, en charge de la gestion du groupe
        \item Mamene-Mokosson Eymard : Assistant chef de projet, en charge des dossiers administratifs (cahier des charges, rapport etc)
		 
        \item Bazia Vincent, chef du developpement, en charge de la repartition et gestion de la programmation du projet
        
\end{itemize}

Ainsi, en fonction de nos rôles et des besoins, nous définissons les tâches à faire puis nous les affectons à nos deux autres camarades.

Pour une meilleure efficacité et communication, nous nous rencontrions toutes les semaines voire un peu plus, soit en ligne sur Facebook, Teamspeak ou Skype, soit en physique à la fac où ailleurs pour discuter des avancées et fixer les tâches restantes à réaliser.

De plus, nous rencontrions notre tuteur de projet, Mr Lafourcade , environ toutes les deux semaines pour lui presenter les avancées du projet, avoir ses impressions et discuter des améliorations ou définir de nouvelles orientations au niveau de la programmation ou des fonctionnalités.\vspace{.07in}

Par ailleurs, ne maitrisant pas tous, toutes les technologies à deployer pour le projet notamment node.js, nous avons pris l'habitude au début de programmer tous ensemble dans une même pièce afin de d'accélérer le proecessus d'apprentissage. Puis par la suite, une fois à l'aise, nous sommes sommes orientés vers une programmation en solitaire avec utilisation d'outils de versionnage et de partage de code comme github. 


\section{Les outils utilisés}

----------------- à faire  ---------------------------






\section{Planification}

Le projet à commencé le 11 février et s'est terminé le 19 mars de l'année 2015.

Nous avons réalisé un planning prévisionnel au début du projet afin de pouvoir réaliser un suivi de notre avancement et bien se repartir le travail. Auparavant, nous avons défini les grandes taches de notre projet qui sont la conception du jeu, le développement et l'amélioration des performances réseau. De plus nous nous sommes réparti le travail en fonction de nos préférences mais, en faisant en sorte que tout le monde travail plus au moins sur chaques parties.

Nous avions planifié le projet pour quatre personnes, mais l'un des membres de l'équipe n'a jamais travaillé sur le projet. Lorsque nous avons appris que nous n'étions plus que trois, nous nous sommes réparti les tâches qui devaient réaliser par le quatrième membre. Certaines parties ont donc pris plus de temps que prévu.

Le diagramme prévisionnel au lancement du projet est le suivant :
------------
mettre diagramme openproj ici (ou en annexe si trop gros)
------------

Malgré l'absence d'un des membres, nous avons réussi à suivre nos délais sur le projet global, mais pour certaines taches, nous avons eu des retards. En effet, une session de rattrapage de partiels était organisée de 31 Mars au 6 Avril. Les membres du groupe étant convoqué à cette session, nous avons marqué une pause dans l'avancement du projet pour nous consacrer à la préparation de cet examen.

Ce qui nous donne le planning suivant :
------------
mettre diagramme openproj réel ici (ou en annexe si trop gros)
------------

Des taches ont durée moins longtemps que prévu comme par exemple la création de l'architecture du projet qui a pris peu de temps, ou bien le développement du client et du serveur que l'on avait évalué comme une tache longue, mais qui au final, malgré la pause révision, a été terminé dans les temps.

Des rendez-vous étaient régulièrement fixé avec le tuteur de projet M. Lafourcade afin de se mettre d'accord sur l'avancement du projet et les corrections ou amélioration qu'il fallait apporter au jeu. Ces rendez-vous se faisaient au rythme de 1 rendez-vous toutes les 2 semaines.