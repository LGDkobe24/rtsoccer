\chapter{Introduction}

Le jeu de stratégie en temps réel (RTS pour la dénomination du genre en anglais : real-time strategy) est un type de jeu de stratégie particulier qui notamment et par opposition au jeu de stratégie au tour par tour n’utilise pas un découpage arbitraire du temps en tours de jeu.\\
Pour comprendre le temps réel, il convient de décrire le système au tour par tour puis d’en expliquer les différences. Ainsi dans le tour par tour, l’issue d’une confrontation est résolue par étapes, un combat en succédant un autre, de manière à laisser à chaque joueur le temps de réfléchir sur la prochaine étape.Mais le problème qui se pose lorsqu’on veut simuler des affrontements réalistes est qu’en général, les unités n’attendent pas que ce soit leur tour pour attaquer et gérer l’affrontement de plusieurs unités en même temps et de façon réaliste relève du domaine de l’impossible pour l’être humain, jusqu’à l’invention des processeurs. Ceux-ci, programmés de manière à simuler un affrontement peuvent désormais gérer le déplacement de plusieurs unités simultanément et résoudre les combats de façon simultanée et rigoureuse dans le cadre d’un jeu vidéo.

Une arène de bataille en ligne multijoueur (en anglais, Multiplayer online battle arena : MOBA) est un type de jeu vidéo qui tire son origine de Defense of the Ancients (DotA), un mod de Warcraft III: Reign of Chaos sorti en 2003, associant le jeu de stratégie en temps réel et le jeu de rôle.\\

Le but de notre projet est de developper un jeu, nommé RTSoccer, qui mélange football et RTS. Ce jeu va utiliser les mécanismes du MOBA et être jouable en ligne sur un navigateur. 

Dans un premier temps, nous présenterons le projet à travers le cahier des charges, puis nous traiterons le rapport technique où sera expliquée la phase de conception. Enfin le rapport d’activité relatera les moyens et les outils mis en œuvre pour réaliser le projet.

